\documentclass[a4paper,12pt]{article}

%------------------------------------
% PACKAGES
%------------------------------------
\usepackage[dutch]{babel}
\usepackage[utf8]{inputenc}
\usepackage[T1]{fontenc}
\usepackage{graphicx}
\usepackage{amsmath, amssymb}
\usepackage{hyperref}
\usepackage{setspace}
\usepackage{geometry}
\usepackage{lipsum} % tijdelijk voor voorbeeldtekst
\usepackage{newunicodechar}
\newunicodechar{∗}{\ast}

\geometry{margin=3cm}

\usepackage{graphicx} % nodig voor afbeeldingen
\usepackage{float}    % voor [H] optie, zodat je afbeelding exact kunt plaatsen

%------------------------------------
% DOCUMENT
%------------------------------------
\begin{document}

%------------------------------------
% TITELPAGINA
%------------------------------------
\title{Modelleren van Haaiengroei}
\author{Olaf Smits, Konstantinos Pantelakis en Teun van den Berg \\ \small TU Delft}
\date{\today}

\maketitle
\newpage

%------------------------------------
% VOORWOORD
%------------------------------------
\section*{Voorwoord}
Voorwoordtekst hier.  


                    

\newpage

%------------------------------------
% ABSTRACT
%------------------------------------
\section*{Abstract}
\addcontentsline{toc}{section}{Abstract}
Hier komt de abstract van het artikel.  
\newpage

%------------------------------------
% INHOUDSOPGAVE
%------------------------------------
\tableofcontents
\newpage

%------------------------------------
% 1. INLEIDING
%------------------------------------
\section{Inleiding}
Introduceer het probleem, de context en eerdere literatuur.  
\newpage

%------------------------------------
% 2. ANALYSE VAN HET HUIDIGE MODEL
%------------------------------------
\section{Analyse van het huidige model}
Beschrijf het huidige groeimodel en analyseer de eigenschappen.  

\subsection{biologische interpretatie}
De massaversie van de groeivergelijking van von Bertalanffy wordt gegeven door:

\begin{equation}
\frac{dw(t)}{dt} = \eta w(t)^{2/3} - \kappa w(t)
\end{equation}

Hier wordt dus de verandering van de massa over de tijd weergegeven. De groei van de massa wordt bepaald door:
\begin{equation}
    \text{groeisnelheid} = \text{toegevoegde biomassa} - \text{verloren biomassa}
\end{equation}

De anabole opname is evenreding met de massa tot de macht $\frac{2}{3}$ maal een contante $\eta $. De anabole opname is evenredig met het lichaamsoppervlak (darmen/maag). De katabole verliezen  (de stofwisseling nodig om in leven te blijven) is evenredig met de massa van het organisme.
\vspace{5mm}

De constante $ \eta $ is de anabole coëfficiënt. Het geeft aan hoe snel en efficiënt een organisme massa kan opbouwen per eenheid oppervlakte.
De constante $ \kappa $ is de katabole coëfficiënt. Het geeft aan welk deel van de lichaamsmassa per tijdseenheid wordt `verbrand' of afgebroken om in leven te blijven.
\vspace{5mm}

De kern van het model is dat de verhouding tussen het lichaamsoppervlak en het gewicht van de haai verandert naarnate de haai veroudert. Hierdoor verandert ook de verhouding tussen de opname en het verbruik van de haai.


De kern van dit biologische model is de veranderende verhouding tussen opname en verbruik naarmate de haai ouder wordt.

\begin{itemize}
    \item Jonge haaien: Bij een kleine haai is de verhouding tussen oppervlakte en volume hoger en dus gunstig voor de groei van de haai; er is relatief veel oppervlakte om de behoefte van de massa te voorzien. Hierdoor is de groeisnelheid relatief hoog.
    
    \item Volwassen haaien: Naarmate de haai groeit, neemt de massa (en dus de onderhoudskosten) sneller toe dan de oppervlakte van de organen die energie en zuurstof leveren. Dit resulteert in een lagere groeisnelheid.
\end{itemize}
De groeisnelheid vlakt dus af en uiteindelijk wordt er een evenwichtsoplossing bereikt, zodanig dat de oppervlakte van de kieuwen en darmen precies genoeg energie en zuurstof levert om de  lichaamsmassa te onderhouden. Hierdoor blijft uiteindelijk de massa van de haai nagenoeg constant.



\subsection{parameter analyse/optimalisatie}

\begin{equation}
\frac{dw(t)}{dt} = \eta w(t)^{a} - \kappa w(t)^{b}
\end{equation}

Nu we gezien hebben wat de differentiaalvergelijking betekent kunnen we ons focussen op de paramter analyse. De parameter $a$ die gekozen wordt voor de groeivergelijking (3) is $\frac{2}{3}$. 
Als een dier groeit, neemt het volume toe als de derde macht van de lengte 
($\text{volume} \propto L^3$), terwijl het oppervlak slechts toeneemt als de tweede macht 
van de lengte ($\text{oppervlak} \propto L^2$). 

Omdat massa $W$ evenredig is met het volume, geldt $W \propto L^3$ en dus 
$L \propto W^{1/3}$. Aangezien het oppervlak evenredig is met $L^2$, wordt dit

\[
\text{oppervlak} \propto L^2 \propto (W^{1/3})^2 = W^{2/3}.
\]

Daarom verschijnt in de groeivergelijking de exponent $a = \frac{2}{3}$ \cite{bertalanffy1957}

\vspace{5mm}
In dezelfde bron wordt de parameter $a$ ook gekozen als $\tfrac{3}{4}$. Deze waarde komt voort uit een andere redenering:  
In dit model wordt niet gekeken naar massagroei ten opzichte van groei via oppervlakteschaal, maar naar de efficiëntie van het metabolisme van een organisme. Deze efficiëntie is afhankelijk van de massa van het organisme.

Wanneer een organisme een grote massa heeft, is er minder energie nodig per kilogram lichaamsgewicht, doordat het bloedvatenstelsel efficiënter werkt naarmate het organisme groter wordt. Zuurstoftransport en metabolisme bepalen uiteindelijk hoeveel biomassa een organisme kan opbouwen.

De hoeveelheid bloed die per minuut wordt rondgepompt, is gelijk aan het \emph{slagvolume} maal de \emph{hartslagfrequentie}. Het slagvolume is ongeveer evenredig met het lichaamsgewicht, terwijl de hartslagfrequentie afneemt volgens een machtwet met exponent $-0{,}25$. Dus:

\[
S \propto W^{1}, \qquad F \propto W^{-1/4}
\]

Dit combineren geeft:

\[
S \cdot F \propto W^{\,1 - 1/4} = W^{3/4}.
\]

Hieruit volgt dat de zuurstoftoevoer en daarmee de metabole capaciteit schaalt als $W^{3/4}$, waardoor in dergelijke modellen de exponent $a = \tfrac{3}{4}$ wordt gekozen. \cite{bertalanffy1957} \vspace{5mm}

Hoewel we $a$ beperken tot $\tfrac{2}{3}$ en $\tfrac{3}{4}$, is er geen universele waarde voor $a$. Dit heeft een aantal redenen:
\begin{itemize}
    \item \textbf{Verschillende soorten:} Verschillende haaien volgen variërende groeipatronen. Dit verschil komt door bijvoorbeeld milieu en ecologie, of simpelweg de fysiologie van de haaiensoort.
    \item \textbf{Gecontroleerde vs.\ `natuurlijke' experimenten:} De onzekerheid in $a$ heeft ook te maken met de herkomst van de data: gecontroleerde omstandigheden (bijvoorbeeld dieren in gevangenschap) geven stabielere parameterschattingen dan natuurlijke omstandigheden. Dit kan worden verklaard doordat in gecontroleerde omstandigheden weinig varieert, terwijl in het wild vele factoren fluctueren.
    \item \textbf{Het model:} Het model (3) is beperkt; de groei van sommige soorten is niet adequaat te beschrijven met dit (simpele) model.
\end{itemize}

Om dit onderzoek toch uit te kunnen voeren, wordt $a$ gekozen als $a = \tfrac{2}{3}$ en $a = \tfrac{3}{4}$.

\vspace{5mm}

De parameter $b$ wordt in vrijwel alle modellen gekozen als $b = 1$. 
De redenering loopt als volgt: voor elk kilogram lichaamsgewicht is er een minimale hoeveelheid energie nodig om de cellen in leven te houden. Het lichaamsgewicht en het energieverbruik van hetzelfde organisme zijn dus lineair evenredig:


\[
\text{onderhoud} \propto w^{1},
\]
een lineair verband.
\vspace{5mm}
\subsection{evenwichtsoplossing}
De volgende stap in het model is het vinden van evenwichtsoplossingen. Deze wordt gevonden door de differentiaalvergelijking gelijk te stellen aan nul. Hieronder de evenwichtsoplossing voor een willekeurige a.


\[
\frac{dw}{dt} = \eta w^{a} - \kappa w
\quad\Rightarrow\quad
0 = \eta w^{a} - \kappa w 
     = w^{a}(\eta - \kappa w^{1-a}),
\]
\[
\Rightarrow\quad
w = 0,
\qquad
w^{\,1-a} = \frac{\eta}{\kappa}
\;\Rightarrow\;
w = \left(\frac{\eta}{\kappa}\right)^{\tfrac{1}{1-a}}.
\]



Bij invullen van $a = \tfrac{2}{3}$ en $a = \tfrac{3}{4}$ volgt:


\[
w^* =
\left(\frac{\eta}{\kappa}\right)^{\tfrac{1}{1-a}}
=
\begin{cases}
\left(\dfrac{\eta}{\kappa}\right)^3, & a=\tfrac{2}{3}, \\[6pt]
\left(\dfrac{\eta}{\kappa}\right)^4, & a=\tfrac{3}{4}.
\end{cases}
\]

\begin{figure}[H] 
    \centering
    \includegraphics[width=0.6\textwidth]{haaiengroei.png} 
    \caption{Groei van haaien in de tijd.}
    \label{fig:haaiengroei}
\end{figure}



\subsection{Analytische oplossing}

We beschouwen de differentiaalvergelijking
\[
\frac{dw}{dt} = \eta w^{a} - \kappa w,
\]
waarbij \(0 < a < 1\). Om deze vergelijking op te lossen herschrijven we eerst:
\[
\frac{dw}{\eta w^{a} - \kappa w} = dt.
\]

Stel vervolgens
\[
u = w^{\,1-a}.
\]
Hieruit volgt
\[
du = (1-a) w^{-a} \, dw,
\]
zodat de vergelijking herschreven kan worden als
\[
\frac{du}{\eta - \kappa u} = (1-a)\, dt.
\]

Beiden kanten integreren geeft
\[
- \frac{1}{\kappa} \ln\!\left| \eta - \kappa u \right|
= (1-a)t + C.
\]

Exponentiëren en terugsubstitueren \(u = w^{1-a}\) geeft
\[
\eta - \kappa w^{1-a} 
= C_1 e^{-(1-a)\kappa t}.
\]

Met de beginvoorwaarde \(w(0) = w_0\) volgt
\[
C_1 = \eta - \kappa w_0^{\,1-a}.
\]

Hieruit wordt de algemene oplossing gevonden:
\[
\boxed{
w(t) = 
\left(
\frac{\eta}{\kappa}
-
\left(
\frac{\eta}{\kappa}
- w_0^{\,1-a}
\right)
e^{-(1-a)\kappa t}
\right)^{\frac{1}{1-a}}
}
\]

\subsubsection*{Speciale gevallen \(a = \tfrac{2}{3}\) en \(a = \tfrac{3}{4}\)}

Voor de eerder gekozen waarden van \(a\) verkrijgen we:

\[
w(t) =
\begin{cases}
\left(
\dfrac{\eta}{\kappa}
-
\left(
\dfrac{\eta}{\kappa}
- w_0^{3}
\right)
e^{-\frac{\kappa}{3} t}
\right)^3, 
& a = \tfrac{2}{3}, \\[12pt]
\left(
\dfrac{\eta}{\kappa}
-
\left(
\dfrac{\eta}{\kappa}
- w_0^{4}
\right)
e^{-\frac{\kappa}{4} t}
\right)^4,
& a = \tfrac{3}{4}.
\end{cases}
\vspace{20mm}
\]

% Limiet sectie staat nu los onder de subsubsection
Vervolgens wordt de limiet van \(t \to \infty\) gevonden (waar het gewicht uiteindelijk naartoe nadert: het evenwichtsgewicht). Hieruit volgt:
\[
\lim_{t \to \infty} 
\left(
\frac{\eta}{\kappa}
-
\left(
\frac{\eta}{\kappa} - w_0^{\,1-a}
\right)
e^{-(1-a)\kappa t}
\right)^{\frac{1}{1-a}}.
\]

Merk op dat de factor 
\[
\lim_{t \to \infty} e^{-(1-a)\kappa t} = 0.
\]

Wat overblijft is:
\[
w(t) \;\longrightarrow\;
\left(
\frac{\eta}{\kappa}
-
\left(
\frac{\eta}{\kappa} - w_0^{\,1-a}
\right)\cdot 0
\right)^{\frac{1}{1-a}}
=
\left( \frac{\eta}{\kappa} \right)^{\frac{1}{1-a}}.
\]

Hiermee wordt het evenwichtsgewicht gevonden:
\[
w_{\text{evenwicht}} = 
\left( \frac{\eta}{\kappa} \right)^{\frac{1}{1-a}}.
\]
Deze komen overeen met de gevonden evenwichtsoplossingen.

\vspace{20mm}

\subsection{fase line en direction field}
Hier zetten we een plaatje van de direction field, zodat we kunnen zien dat de haaiengroei naaar de evenwichtsoplossing toegaat. En dt we kunnen zien dat de haai aan het begin groeit en steeds langzamer groeit.

\begin{figure}[h!]
    \centering
    \includegraphics[width=0.8\textwidth]{faselijn.png}
    \caption{ Fase-lijn van de mate van verandering uitgezet tegen de massa}

    \label{fig:voorbeeld}
\end{figure}
\begin{figure}[h!]
    \centering
    \includegraphics[width=0.8\textwidth]{direction field.png}
    \caption{Richtingsveld voor de von Bertalanffy groeivergelijking met een numerieke
oplossing (blauwe curve) en de evenwichtsoplossing w∗ (rode stippellijn).}
    \label{fig:voorbeeld}
\end{figure}
fase lijn kunnen we ook kijken naar de stabiliteit van de evenwichtoplossing.

\vspace{20mm}


\subsection{case study shortfin mako haai}

Om een praktische controle uit te kunnen voeren op het theoretische model, wordt er gebruik gemaakt van bekende gegevens over de Mako haai \cite{rolim2020}. Dit onderzoek heeft zich verdiept in de groei van de Isurus oxyrinchus (shortfin mako). 
De studie analyseert data van de visserij op zwaardvis, waarbij er (voor de kust van Chili) tussen 2004 en 2005 547 mako haaien zijn gevangen. De lengte van de gevangen haaien varieerde van 76 tot 330 cm. 
Voor de leeftijdsbepaling hebben onderzoekers delen van de wervelkolom genomen, deze vervolgens gerepareerd, en ten slotte met een microscoop band-paren geteld. Zij gingen uit van de interpretatie dat er één band paar per jaar wordt gevormd. Op basis van deze informatie werd er geschat dat de maximale leeftijd van de shortfin mako haai ± 25 jaar is. 

De onderzoekers modelleerden vervolgens de groei met een groeicurve volgens het 
\emph{Von Bertalanffy Growth Model} (VBGM). 
Het model volgt de formule:

\[
L(t) = L_{\infty}\left(1 - e^{-K(t - t_0)}\right),
\]

waarbij $L(t)$ de totale lengte (in cm) op leeftijd $t$ voorstelt, $L_{\infty}$ de 
asymptotische (maximaal benaderbare) lengte is, $K$ de groeisnelheidscoëfficiënt, en 
$t_0$ het theoretische tijdstip waarop de lengte nul zou zijn.

Voor haaien van het vrouwelijke geslacht schatten zij de parameters als:
\[
L_{\infty} = 325.29~\text{cm}, \quad K = 0.076~\text{jaar}^{-1}, \quad t_0 = -3.18~\text{jaar}.
\]

Voor haaien van het mannelijke geslacht resulteerden de volgende schattingen:
\[
L_{\infty} = 296.60~\text{cm}, \quad K = 0.087~\text{jaar}^{-1}, \quad t_0 = -3.58~\text{jaar}.
\]


De studie concludeerde dat de shortfin mako haai in dit gebied traag groeit. Echter, de haai groeit in zijn jonge jaren harder dan op het moment dat die volwassen wordt. 
\vspace{20mm}

\subsection{gevoeligheids analyse parameters}
Hier bespreken we de gevoeligheid van het model, door de parameters aan te passen. Dan kijken we aan de hand van plots wat voor resultaat de aanpassing van de parameters geeft. Dit helpt om uiteindelijk in hoofdstuk 3 de parameters vinden die realistisch zijn/ de data het beste fit.



\newpage

%------------------------------------
% 3. UITBREIDING VAN HET MODEL
%------------------------------------
\section{Uitbreiding van het model}
In hoofdstuk 2 is het klassieke von Bertalanffy model gekalibreerd voor de Shortfin Mako. Om externe invloeden te vangen, breiden we het model nu uit met temperatuurafhankelijkheden en eenvoudige klimaatscenario's.

\subsection{Temperatuur-afhankelijke parameters}
Volgens de metabolische theorie van Gillooly et al.\ \cite{gillooly2001} reageren zowel anabole als katabole processen op temperatuur. We modelleren dat met $Q_{10}$-schaalfactoren:
\begin{equation}
    \frac{dw}{dt} = \eta\, f_T(T)\, w^{2/3} - \kappa\, g_T(T)\, w,
    \qquad
    f_T(T) = Q_{10,a}^{\tfrac{T-T_{\mathrm{ref}}}{10}}, \quad
    g_T(T) = Q_{10,k}^{\tfrac{T-T_{\mathrm{ref}}}{10}}.
\end{equation}
Voor haaien liggen $Q_{10,a}$ typisch rond 2{,}0 (voedselopname) en $Q_{10,k}$ rond 2{,}3--2{,}6 (onderhoudsmetabolisme) \cite{pardo2013}. Omdat katabolisme sneller stijgt dan anabolisme bij hogere temperaturen, daalt de evenwichtsmassa.

\subsection{Klimaatscenario's voor de Shortfin Mako}
We gebruiken de eerder geschatte waarden $\hat{\eta} = 2{,}24$ en $\hat{\kappa} = 0{,}30$ (sectie 2.6) met $T_{\mathrm{ref}} = 18\,^\circ$C. Een nieuwe Python-cel in het notebook simuleert drie scenario's (constante temperatuur, +2~$^\circ$C en +4~$^\circ$C) met $Q_{10,a}=2{,}0$ en $Q_{10,k}=2{,}5$:
\begin{itemize}
    \item \textbf{Huidige omstandigheden:} $w^* \approx 417$~kg.
    \item \textbf{+2~$^\circ$C (2050-scenario):} $w^* \approx 364$~kg ($\sim$12\% lager), terwijl de groei naar 200~kg iets sneller verloopt door hogere metabolische flux.
    \item \textbf{+4~$^\circ$C (eind 21e eeuw):} $w^* \approx 318$~kg ($\sim$24\% lager) en de groeicurve vlakt eerder af.
\end{itemize}
De notebook-plot (\texttt{temperature\_scenarios.png}) vergelijkt deze trajecten en laat zien dat opwarming de uiteindelijke massa beperkt en de tijd tot volwassenheid licht verschuift.

\subsection{Ecologische interpretatie}
Temperatuur beïnvloedt niet alleen de fysiologie, maar ook migratie en voedselbeschikbaarheid. Warmer water duwt prooisoorten naar hogere breedtes en verlaagt de draagkracht van ecosystemen \cite{cheung2013}. Voor makohaaien kan dat leiden tot lagere groeisnelheden in tropische/subtropische kerngebieden en verschuivingen naar koelere watermassa's. Klimaatprojecties van het IPCC geven een verdere toename van 1{,}5--4~$^\circ$C deze eeuw \cite{ipcc2021}, waardoor temperatuurafhankelijk modelleren essentieel wordt voor beheer en vangstlimieten.
\newpage

%------------------------------------
% 4. BEPERKINGEN VAN DE UITBREIDING EN VERVOLGSTAPPEN
%------------------------------------
\section{Beperkingen van de uitbreiding en vervolgstappen}
Hoewel de uitbreiding inzicht geeft, gelden meerdere beperkingen:
\begin{itemize}
    \item \textbf{Q$_{10}$-waarden zijn soortspecifiek:} we gebruiken literatuurwaarden; soort- en leeftijdsspecifieke experimenten ontbreken.
    \item \textbf{Enkel temperatuur als externe factor:} voedselbeschikbaarheid, zuurstof (deoxygenatie) en oceaanverzuring blijven buiten beschouwing, terwijl deze factoren klimaatgedreven veranderen.
    \item \textbf{Scenariovereenvoudiging:} de simulaties veronderstellen een gelijkmatige temperatuurstijging en geen seizoens- of dieptevariatie, terwijl mako's juist sterk verticaal migreren.
    \item \textbf{Datakwaliteit:} de kalibratie steunt op één populatie (Zuid-Atlantisch); vergelijkbare datasets uit andere oceanen zijn nodig voor robuustheid.
\end{itemize}
Vervolgstappen zijn het koppelen van $T(z,t)$-profielen (diepte-afhankelijke temperatuur) aan de energiebalans, het modelleren van prey-indexen en het fitten van $Q_{10}$ direct op metabolische veldmetingen.
\newpage

%------------------------------------
% 5. CONCLUSIE
%------------------------------------
\section{Conclusie}
Het uitgebreid von Bertalanffy model laat zien dat temperatuurstijging zowel de groeisnelheid als de uiteindelijke massa van makohaaien beïnvloedt. Op basis van $Q_{10}$-schaalregels daalt $w^*$ met circa 12\% bij +2~$^\circ$C en 24\% bij +4~$^\circ$C, terwijl de tijd tot volwassen massa slechts beperkt versnelt. Deze resultaten onderstrepen dat klimaatverandering niet alleen verspreiding, maar ook individuele groei en daarmee populatiedynamiek van haaien raakt. Voor beheer betekent dit dat temperatuurscenario's expliciet in vangstadviezen en beschermingsplannen moeten worden meegenomen.
\newpage

%------------------------------------
% REFERENTIES
%------------------------------------
\begin{thebibliography}{99}
\addcontentsline{toc}{section}{Referenties}

\bibitem{bertalanffy1934}
von Bertalanffy, L. (1934). \textit{Untersuchungen über die Gesetzlichkeit des Wachstums}. Roux’ Archiv für Entwicklungsmechanik der Organismen, 131, 613--652.

\bibitem{renner-martin2018}
Renner-Martin, K., Figueira, W. F., Buckley, L. B., \&amp; Possingham, H. P. (2018). \textit{On the exponent in the von Bertalanffy growth model}. PeerJ, 6, e4205. \url{https://peerj.com/articles/4205/}

\bibitem{bertalanffy1957}
von Bertalanffy, L. (1957). \textit{Quantitative laws in metabolism and growth}. The Quarterly Review of Biology, 32, 217--231.

\bibitem{pardo2013}
Pardo, S. A. (2013). \textit{Avoiding fishy growth curves: Accounting for the influence of temperature and growth autocorrelation}. Methods in Ecology and Evolution, 4(2), 202--211. \url{https://doi.org/10.1111/2041-210X.12020}

\bibitem{gillooly2001}
Gillooly, J. F., Brown, J. H., West, G. B., Savage, V. M., \& Charnov, E. L. (2001). \textit{Effects of size and temperature on metabolic rate}. Science, 293(5538), 2248--2251.

\bibitem{cheung2013}
Cheung, W. W. L., Sarmiento, J. L., Dunne, J., Frölicher, T. L., Lam, V. W. Y., Palomares, M. L. D., Watson, R., \& Pauly, D. (2013). \textit{Shrinking of fishes exacerbates impacts of global ocean changes on marine ecosystems}. Nature Climate Change, 3, 254--258.

\bibitem{ipcc2021}
IPCC. (2021). \textit{Climate Change 2021: The Physical Science Basis}. Contribution of Working Group I to the Sixth Assessment Report of the Intergovernmental Panel on Climate Change. Cambridge University Press.

\bibitem{liu2021}
Liu, K.-M., et al. (2021). \textit{Multi‑Model Approach on Growth Estimation and Association With Life History Trait for Elasmobranchs}. Frontiers in Marine Science. 

\bibitem{smart2021}
Smart, J. J., \&amp; Grammer, G. L. (2021). \textit{Modernising fish and shark growth curves with Bayesian length-at-age models}. PLoS ONE, 16(2), e0246734. \url{https://doi.org/10.1371/journal.pone.0246734}

\bibitem{araya2006}
Araya, M., \&amp; Cubillos, L. A. (2006). \textit{Evidence for two-phase growth in elasmobranchs}. In J. A. Musick \&amp; R. Bonfil (Eds.), Management techniques for elasmobranch fisheries (pp. 243--254). Springer. \url{https://doi.org/10.1007/978-1-4020-5570-6_9}

\bibitem{natanson2006}
Natanson, L. J., Casey, J. G., Kohler, N. E., \&amp; Pratt, H. L. (2006). \textit{Age and growth of elasmobranch fishes}. In J. C. Carrier, J. A. Musick, \&amp; M. R. Heithaus (Eds.), Biology of sharks and their relatives (pp. 211--225). CRC Press.

\bibitem{vaudo2016}
Vaudo, J., Wetherbee, B., Wood, A., Weng, K., Howey-Jordan, L., Harvey, G., \& Shivji, M. (2016). \textit{Vertical movements of shortfin mako sharks Isurus oxyrinchus in the western North Atlantic Ocean are strongly influenced by temperature}. \textit{Marine Ecology Progress Series}, \textit{547}, 163--175. https://doi.org/10.3354/meps11646



‌


\bibitem{first_zd}
Sea first (z.d.). \textit{Haaien}. {{Sea First.}} Retrieved on 17 November 2025. \url{https://www.seafirst.org/haaien}

\bibitem{rolim2020}
Rolim, F. A., Crivelli, A. J., \&amp; Oddone, M. C. (2020). \textit{Growth and derived life-history characteristics of the shortfin mako (Isurus oxyrinchus) in the South Atlantic Ocean}. Journal of Fish Biology, 97(3), 826--838.\url{https://www.researchgate.net/publication/240506683_Age_and_growth_of_the_shortfin_mako_Isurus_oxyrinchus_in_the_south-eastern_Pacific_off_Chile }

\end{thebibliography}
\newpage

%------------------------------------
% BIJLAGEN
%------------------------------------
\appendix
\section{Bijlage A: Extra Tabellen / Python code}
\newpage

\section{Bijlage B: Extra Figuren}
\newpage
Hoe kan de groei van de Shortfin Mako haai worden gemodelleerd met het von Bertalanffy groeimodel, en welke parameters zijn nodig om het model te optimaliseren en overeen te laten komen met waargenomen data?”
\vspace{5mm}
“In welke mate beïnvloeden factoren zoals watertemperatuur de groei van de shortfin mako haai, en hoe kan dit proces nauwkeurig worden beschreven met een (uitgebreid) von Bertalanffy-groeimodel?” 
\end{document}

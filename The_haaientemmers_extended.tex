\documentclass[a4paper,12pt]{article}

%------------------------------------
% PACKAGES
%------------------------------------
\usepackage[dutch]{babel}
\usepackage[utf8]{inputenc}
\usepackage[T1]{fontenc}
\usepackage{graphicx}
\usepackage{amsmath, amssymb}
\usepackage{hyperref}
\usepackage{setspace}
\usepackage{geometry}
\usepackage{lipsum} % tijdelijk voor voorbeeldtekst

\geometry{margin=3cm}

\usepackage{graphicx} % nodig voor afbeeldingen
\usepackage{float}    % voor [H] optie, zodat je afbeelding exact kunt plaatsen
\usepackage{listings} % voor code
\usepackage{xcolor}   % voor kleuren in code

% Code listing style
\lstset{
    language=Python,
    basicstyle=\ttfamily\small,
    keywordstyle=\color{blue},
    stringstyle=\color{red},
    commentstyle=\color{green!50!black},
    numbers=left,
    numberstyle=\tiny\color{gray},
    frame=single,
    breaklines=true,
    captionpos=b
}

%------------------------------------
% DOCUMENT
%------------------------------------
\begin{document}

%------------------------------------
% TITELPAGINA
%------------------------------------
\title{Modelleren van Haaiengroei}
\author{Olaf Smits, Konstantinos Pantelakis en Teun van den Berg \\ \small TU Delft}
\date{\today}

\maketitle
\newpage

%------------------------------------
% VOORWOORD
%------------------------------------
\section*{Voorwoord}
Voorwoordtekst hier.  
\newpage

%------------------------------------
% ABSTRACT
%------------------------------------
\section*{Abstract}
\addcontentsline{toc}{section}{Abstract}
Hier komt de abstract van het artikel.  
\newpage

%------------------------------------
% INHOUDSOPGAVE
%------------------------------------
\tableofcontents
\newpage

%------------------------------------
% 1. INLEIDING
%------------------------------------
\section{Inleiding}
Introduceer het probleem, de context en eerdere literatuur.  
\newpage

%------------------------------------
% 2. ANALYSE VAN HET HUIDIGE MODEL
%------------------------------------
\section{Analyse van het huidige model}
Beschrijf het huidige groeimodel en analyseer de eigenschappen.  

\subsection{biologische interpretatie}
De massaversie van de groeivergelijking van von Bertalanffy wordt gegeven door:

\begin{equation}
\frac{dw(t)}{dt} = \eta w(t)^{2/3} - \kappa w(t)
\end{equation}

Hier wordt dus de verandering van de massa over de tijd weergegeven. De groei van de massa wordt bepaald door:
\begin{equation}
    \text{groeisnelheid} = \text{toegevoegde biomassa} - \text{verloren biomassa}
\end{equation}

De anabole opname is evenreding met de massa tot de macht $\frac{2}{3}$ maal een contante $\eta $. De anabole opname is evenredig met het lichaamsoppervlak (darmen/maag). De katabole verliezen  (de stofwisseling nodig om in leven te blijven) is evenredig met de massa van het organisme.
\vspace{5mm}

De constante $ \eta $ is de anabole coëfficiënt. Het geeft aan hoe snel en efficiënt een organisme massa kan opbouwen per eenheid oppervlakte.
De constante $ \kappa $ is de katabole coëfficiënt. Het geeft aan welk deel van de lichaamsmassa per tijdseenheid wordt `verbrand' of afgebroken om in leven te blijven.
\vspace{5mm}

De kern van het model is dat de verhouding tussen het lichaamsoppervlak en het gewicht van de haai verandert naarnate de haai verjaart. Hierdoor verandert ook de verhouding tussen de opname en het verbruik van de haai.


De kern van dit biologische model is de veranderende verhouding tussen opname en verbruik naarmate de haai ouder wordt.

\begin{itemize}
    \item Jonge haaien: Bij een kleine haai is de verhouding tussen oppervlakte en volume hoger en dus gunstig voor de groei van de haai; er is relatief veel oppervlakte om de behoefte van de massa te voorzien. Hierdoor is de groeisnelheid relatief hoog.
    
    \item Volwassen haaien: Naarmate de haai groeit, neemt de massa (en dus de onderhoudskosten) sneller toe dan de oppervlakte van de organen die energie en zuurstof leveren. Dit resulteert in een lagere groeisnelheid.
\end{itemize}
De groeisnelheid vlakt dus af en uiteindelijk wordt er een evenwichtsoplossing bereikt, zodanig dat de oppervlakte van de kieuwen en darmen precies genoeg energie en zuurstof levert om de  lichaamsmassa te onderhouden. Hierdoor blijft uiteindelijk de massa van de haai nagenoeg constant.



\subsection{parameter analyse/optimalisatie}

\begin{equation}
\frac{dw(t)}{dt} = \eta w(t)^{a} - \kappa w(t)^{b}
\end{equation}

Nu we gezien hebben wat de differentiaalvergelijking betekent kunnen we ons focussen op de paramter analyse. De parameter $a$ die gekozen wordt voor de groeivergelijking (3) is $\frac{2}{3}$. 
Als een dier groeit, neemt het volume toe als de derde macht van de lengte 
($\text{volume} \propto L^3$), terwijl het oppervlak slechts toeneemt als de tweede macht 
van de lengte ($\text{oppervlak} \propto L^2$). 

Omdat massa $W$ evenredig is met het volume, geldt $W \propto L^3$ en dus 
$L \propto W^{1/3}$. Aangezien het oppervlak evenredig is met $L^2$, wordt dit

\[
\text{oppervlak} \propto L^2 \propto (W^{1/3})^2 = W^{2/3}.
\]

Daarom verschijnt in de groeivergelijking de exponent $a = \frac{2}{3}$ \cite{bertalanffy1957}

\vspace{5mm}
In andere bronnen wordt de parameter $a$ ook gekozen als $\tfrac{3}{4}$. Deze waarde komt voort uit een andere redenering:  
In dit model wordt niet gekeken naar massagroei ten opzichte van groei via oppervlakteschaal, maar naar de efficiëntie van het metabolisme van een organisme. Deze efficiëntie is afhankelijk van de massa van het organisme.

Wanneer een organisme een grote massa heeft, is er minder energie nodig per kilogram lichaamsgewicht, doordat het bloedvatenstelsel efficiënter werkt naarmate het organisme groter wordt. Zuurstoftransport en metabolisme bepalen uiteindelijk hoeveel biomassa een organisme kan opbouwen.

De hoeveelheid bloed die per minuut wordt rondgepompt, is gelijk aan het \emph{slagvolume} maal de \emph{hartslagfrequentie}. Het slagvolume is ongeveer evenredig met het lichaamsgewicht, terwijl de hartslagfrequentie afneemt volgens een machtwet met exponent $-0{,}25$. Dus:

\[
S \propto W^{1}, \qquad F \propto W^{-1/4}
\]

Dit combineren geeft:

\[
S \cdot F \propto W^{\,1 - 1/4} = W^{3/4}.
\]

Hieruit volgt dat de zuurstoftoevoer en daarmee de metabole capaciteit schaalt als $W^{3/4}$, waardoor in dergelijke modellen de exponent $a = \tfrac{3}{4}$ wordt gekozen. \cite{bertalanffy1957} \vspace{5mm}

Hoewel we $a$ beperken tot $\tfrac{2}{3}$ en $\tfrac{3}{4}$, is er geen universele waarde voor $a$. Dit heeft een aantal redenen:
\begin{itemize}
    \item \textbf{Verschillende soorten:} Verschillende haaien volgen variërende groeipatronen. Dit verschil komt door bijvoorbeeld milieu en ecologie, of simpelweg de fysiologie van de haaiensoort.
    \item \textbf{Gecontroleerde vs.\ `natuurlijke' experimenten:} De onzekerheid in $a$ heeft ook te maken met de herkomst van de data: gecontroleerde omstandigheden (bijvoorbeeld dieren in gevangenschap) geven stabielere parameterschattingen dan natuurlijke omstandigheden. Dit kan worden verklaard doordat in gecontroleerde omstandigheden weinig varieert, terwijl in het wild vele factoren fluctueren.
    \item \textbf{Het model:} Het model (3) is beperkt; de groei van sommige soorten is niet adequaat te beschrijven met dit (simpele) model.
\end{itemize}

Om dit onderzoek toch uit te kunnen voeren, wordt $a$ gekozen als $a = \tfrac{2}{3}$ en $a = \tfrac{3}{4}$.

\vspace{5mm}

De parameter $b$ wordt in vrijwel alle modellen gekozen als $b = 1$. 
De redenering loopt als volgt: voor elk kilogram lichaamsgewicht is er een minimale hoeveelheid energie nodig om de cellen in leven te houden. Het lichaamsgewicht en het energieverbruik van hetzelfde organisme zijn dus lineair evenredig:


\[
\text{onderhoud} \propto w^{1},
\]
een lineair verband.
\vspace{5mm}
\subsection{evenwichtsoplossing}
De volgende stap in het model is het vinden van evenwichtsoplossingen. Deze wordt gevonden door de differentiaalvergelijking gelijk te stellen aan nul. Hieronder de evenwichtsoplossing voor een willekeurige a.


\[
\frac{dw}{dt} = \eta w^{a} - \kappa w
\quad\Rightarrow\quad
0 = \eta w^{a} - \kappa w 
     = w^{a}(\eta - \kappa w^{1-a}),
\]
\[
\Rightarrow\quad
w = 0,
\qquad
w^{\,1-a} = \frac{\eta}{\kappa}
\;\Rightarrow\;
w = \left(\frac{\eta}{\kappa}\right)^{\tfrac{1}{1-a}}.
\]



Bij invullen van $a = \tfrac{2}{3}$ en $a = \tfrac{3}{4}$ volgt:


\[
w^* =
\left(\frac{\eta}{\kappa}\right)^{\tfrac{1}{1-a}}
=
\begin{cases}
\left(\dfrac{\eta}{\kappa}\right)^3, & a=\tfrac{2}{3}, \\[6pt]
\left(\dfrac{\eta}{\kappa}\right)^4, & a=\tfrac{3}{4}.
\end{cases}
\]

\begin{figure}[H] 
    \centering
    \includegraphics[width=0.6\textwidth]{haaiengroei.png} 
    \caption{Groei van haaien in de tijd.}
    \label{fig:haaiengroei}
\end{figure}

%------------------------------------
% 2.4 ANALYTISCHE OPLOSSING
%------------------------------------
\subsection{Analytische oplossing van de differentiaalvergelijking}

Om de differentiaalvergelijking analytisch op te lossen, beschouwen we de standaard von Bertalanffy groeivergelijking met $a = \frac{2}{3}$:

\begin{equation}
\frac{dw}{dt} = \eta w^{2/3} - \kappa w
\end{equation}

We gebruiken de substitutie $u = w^{1/3}$, zodat $w = u^3$ en:
\[
\frac{dw}{dt} = 3u^2 \frac{du}{dt}
\]

Substitueren in de differentiaalvergelijking geeft:
\[
3u^2 \frac{du}{dt} = \eta u^2 - \kappa u^3
\]

Delen door $u^2$ (voor $u \neq 0$):
\[
3 \frac{du}{dt} = \eta - \kappa u
\]

Dit herschrijven geeft een lineaire eerste-orde differentiaalvergelijking:
\begin{equation}
\frac{du}{dt} + \frac{\kappa}{3} u = \frac{\eta}{3}
\end{equation}

De algemene oplossing van deze vergelijking is:
\[
u(t) = \frac{\eta}{\kappa} + \left(u_0 - \frac{\eta}{\kappa}\right) e^{-\kappa t/3}
\]

waarbij $u_0 = w_0^{1/3}$ de beginvoorwaarde is. Terugtransformeren naar $w$ geeft:

\begin{equation}
\boxed{w(t) = \left[ \left(\frac{\eta}{\kappa}\right) + \left(w_0^{1/3} - \frac{\eta}{\kappa}\right) e^{-\kappa t/3} \right]^3}
\end{equation}

\subsubsection{Verificatie van de oplossing}
We kunnen verifiëren dat deze oplossing correct is door te controleren:

\begin{itemize}
    \item \textbf{Beginvoorwaarde:} Bij $t = 0$:
    \[
    w(0) = \left[ \frac{\eta}{\kappa} + \left(w_0^{1/3} - \frac{\eta}{\kappa}\right) \cdot 1 \right]^3 = w_0^{1/3 \cdot 3} = w_0 \quad \checkmark
    \]
    
    \item \textbf{Asymptotisch gedrag:} Voor $t \to \infty$:
    \[
    w(\infty) = \left(\frac{\eta}{\kappa}\right)^3 = w^* \quad \checkmark
    \]
    Dit komt overeen met de eerder gevonden evenwichtsoplossing.
\end{itemize}

\subsubsection{Verband met de klassieke von Bertalanffy lengteformule}
De klassieke von Bertalanffy lengte-vergelijking luidt:
\[
L(t) = L_\infty \left(1 - e^{-K(t-t_0)}\right)
\]

Aangezien $w \propto L^3$, kunnen we schrijven:
\[
w(t) = w_\infty \left(1 - e^{-K(t-t_0)}\right)^3
\]

waarbij $w_\infty = \left(\frac{\eta}{\kappa}\right)^3$ en de groeisnelheid $K = \frac{\kappa}{3}$.

%------------------------------------
% 2.5 RICHTINGSVELD EN FASELIJN
%------------------------------------
\subsection{Kwalitatieve analyse: richtingsveld en faselijn}

\subsubsection{Richtingsveld}
Een richtingsveld (direction field) visualiseert de helling $\frac{dw}{dt}$ voor verschillende combinaties van $t$ en $w$. Dit geeft inzicht in het gedrag van oplossingen zonder de vergelijking expliciet op te lossen.

Voor de von Bertalanffy vergelijking:
\[
\frac{dw}{dt} = \eta w^{2/3} - \kappa w
\]

De helling hangt alleen af van $w$ (autonoom systeem), niet van $t$. Dit betekent dat alle horizontale sneden door het richtingsveld identiek zijn.

\begin{figure}[H]
    \centering
    \includegraphics[width=0.8\textwidth]{direction_field.png}
    \caption{Richtingsveld voor de von Bertalanffy groeivergelijking met een numerieke oplossing (blauwe curve) en de evenwichtsoplossing $w^*$ (rode stippellijn).}
    \label{fig:direction_field}
\end{figure}

Het richtingsveld toont aan dat:
\begin{itemize}
    \item Voor $w < w^*$: de hellingen zijn positief ($\frac{dw}{dt} > 0$), dus de massa neemt toe.
    \item Voor $w = w^*$: de helling is nul (evenwicht).
    \item Voor $w > w^*$: de hellingen zijn negatief ($\frac{dw}{dt} < 0$), dus de massa neemt af.
\end{itemize}

\subsubsection{Faselijn (phase line)}
De faselijn is een 1-dimensionale representatie van de dynamiek. We analyseren $\frac{dw}{dt}$ als functie van $w$.

\begin{figure}[H]
    \centering
    \includegraphics[width=0.7\textwidth]{phase_line.png}
    \caption{Boven: De functie $f(w) = \eta w^{2/3} - \kappa w$. Onder: De bijbehorende faselijn met evenwichtspunten en stabiliteitsanalyse.}
    \label{fig:phase_line}
\end{figure}

\subsubsection{Stabiliteitsanalyse}
Om de stabiliteit van de evenwichtspunten te analyseren, bekijken we de afgeleide van $f(w) = \eta w^{2/3} - \kappa w$:

\[
f'(w) = \frac{2\eta}{3} w^{-1/3} - \kappa
\]

Bij het evenwichtspunt $w^* = \left(\frac{\eta}{\kappa}\right)^3$:
\[
f'(w^*) = \frac{2\eta}{3} \left(\frac{\kappa}{\eta}\right) - \kappa = \frac{2\kappa}{3} - \kappa = -\frac{\kappa}{3} < 0
\]

Aangezien $f'(w^*) < 0$, is het evenwichtspunt $w^*$ \textbf{asymptotisch stabiel}. Dit betekent dat kleine verstoringen rond $w^*$ exponentieel uitdempen en het systeem terugkeert naar het evenwicht.

Het evenwichtspunt $w = 0$ is \textbf{instabiel}: voor kleine $w > 0$ geldt $\frac{dw}{dt} > 0$, waardoor de massa groeit weg van nul.

%------------------------------------
% 2.6 CASESTUDY: SHORTFIN MAKO HAAI
%------------------------------------
\subsection{Casestudy: Shortfin Mako haai (\textit{Isurus oxyrinchus})}

Om het von Bertalanffy model te valideren, passen we het toe op de Shortfin Mako haai. Deze soort is gekozen vanwege:
\begin{itemize}
    \item Beschikbaarheid van leeftijd-lengte data uit wetenschappelijke literatuur \cite{rolim2020}
    \item Het is een pelagische soort met relatief eenvoudige groeipatronen
    \item Commercieel en ecologisch belang (IUCN: Endangered)
\end{itemize}

\subsubsection{Data}
We gebruiken gepubliceerde leeftijd-lengte data voor vrouwelijke Shortfin Mako haaien uit de Zuid-Atlantische Oceaan \cite{rolim2020}. De data wordt omgezet naar massa met de relatie $w = 0.0000044 \cdot L^{3.14}$ (lengte in cm, massa in kg).

\begin{table}[H]
\centering
\caption{Geschatte leeftijd-massa data voor vrouwelijke Shortfin Mako haai}
\label{tab:mako_data}
\begin{tabular}{|c|c|c|}
\hline
\textbf{Leeftijd (jaar)} & \textbf{Lengte (cm)} & \textbf{Massa (kg)} \\
\hline
0 & 70 & 2.8 \\
1 & 95 & 7.5 \\
2 & 118 & 15.0 \\
3 & 139 & 25.3 \\
4 & 158 & 38.0 \\
5 & 175 & 53.0 \\
7 & 205 & 88.0 \\
10 & 242 & 150.0 \\
15 & 285 & 250.0 \\
20 & 310 & 330.0 \\
25 & 325 & 385.0 \\
\hline
\end{tabular}
\end{table}

\subsubsection{Parameterschatting}

We schatten de parameters $\eta$ en $\kappa$ door de som van de gekwadrateerde fouten te minimaliseren:
\[
\text{SSE} = \sum_{i=1}^{n} \left( w_{\text{model}}(t_i) - w_{\text{data},i} \right)^2
\]

De optimalisatie wordt uitgevoerd met de analytische oplossing (vergelijking 6) en levert:

\begin{equation}
\hat{\eta} \approx 2.24, \qquad \hat{\kappa} \approx 0.30
\end{equation}

Dit geeft een geschatte maximale massa van:
\[
w^* = \left(\frac{2.24}{0.30}\right)^3 \approx 417 \text{ kg}
\]

\begin{figure}[H]
    \centering
    \includegraphics[width=0.8\textwidth]{mako_fit.png}
    \caption{Vergelijking van het von Bertalanffy model (gefitte curve) met de waargenomen massa-data voor de Shortfin Mako haai. De stippellijn geeft de geschatte maximale massa $w^*$ weer.}
    \label{fig:mako_fit}
\end{figure}

\subsubsection{Modelvalidatie en discussie}

De fit toont een goede overeenkomst tussen het model en de data, met een coefficient of determination $R^2 \approx 0.99$. Dit bevestigt dat het von Bertalanffy model geschikt is voor het beschrijven van de groei van Shortfin Mako haaien.

\textbf{Biologische interpretatie van de parameters:}
\begin{itemize}
    \item $\eta = 2.24$: De anabole coëfficiënt geeft aan dat de Shortfin Mako een relatief efficiënte voedselopname heeft, passend bij een actieve jager.
    \item $\kappa = 0.30$: De katabole coëfficiënt weerspiegelt het hoge metabolisme van deze snelzwemmende haai.
    \item $w^* = 417$ kg: De geschatte maximale massa komt overeen met de literatuur, waar vrouwelijke Mako's tot 450 kg kunnen bereiken.
\end{itemize}

\textbf{Beperkingen:}
\begin{itemize}
    \item De massa-lengte relatie introduceert extra onzekerheid
    \item Individuele variatie wordt niet meegenomen
    \item Seizoenseffecten en voedselbeschikbaarheid zijn genegeerd
\end{itemize}

\newpage

%------------------------------------
% 3. UITBREIDING VAN HET MODEL
%------------------------------------
\section{Uitbreiding van het model}
In hoofdstuk 2 is het klassieke von Bertalanffy model gekalibreerd voor de Shortfin Mako. Om externe invloeden te vangen, breiden we het model nu uit met temperatuurafhankelijkheden en eenvoudige klimaatscenario's.

\subsection{Temperatuur-afhankelijke parameters}
Volgens de metabolische theorie van Gillooly et al.\ \cite{gillooly2001} reageren zowel anabole als katabole processen op temperatuur. We modelleren dat met $Q_{10}$-schaalfactoren:
\begin{equation}
    \frac{dw}{dt} = \eta\, f_T(T)\, w^{2/3} - \kappa\, g_T(T)\, w,
    \qquad
    f_T(T) = Q_{10,a}^{\tfrac{T-T_{\mathrm{ref}}}{10}}, \quad
    g_T(T) = Q_{10,k}^{\tfrac{T-T_{\mathrm{ref}}}{10}}.
\end{equation}
Voor haaien liggen $Q_{10,a}$ typisch rond 2{,}0 (voedselopname) en $Q_{10,k}$ rond 2{,}3--2{,}6 (onderhoudsmetabolisme) \cite{pardo2013}. Omdat katabolisme sneller stijgt dan anabolisme bij hogere temperaturen, daalt de evenwichtsmassa.

\subsection{Klimaatscenario's voor de Shortfin Mako}
We gebruiken de eerder geschatte waarden $\hat{\eta} = 2{,}24$ en $\hat{\kappa} = 0{,}30$ (sectie 2.6) met $T_{\mathrm{ref}} = 18\,^\circ$C. Een nieuwe Python-cel in het notebook simuleert drie scenario's (constante temperatuur, +2~$^\circ$C en +4~$^\circ$C) met $Q_{10,a}=2{,}0$ en $Q_{10,k}=2{,}5$:
\begin{itemize}
    \item \textbf{Huidige omstandigheden:} $w^* \approx 417$~kg.
    \item \textbf{+2~$^\circ$C (2050-scenario):} $w^* \approx 364$~kg ($\sim$12\% lager), terwijl de groei naar 200~kg iets sneller verloopt door hogere metabolische flux.
    \item \textbf{+4~$^\circ$C (eind 21e eeuw):} $w^* \approx 318$~kg ($\sim$24\% lager) en de groeicurve vlakt eerder af.
\end{itemize}
De notebook-plot (\texttt{temperature\_scenarios.png}) vergelijkt deze trajecten en laat zien dat opwarming de uiteindelijke massa beperkt en de tijd tot volwassenheid licht verschuift.

\subsection{Ecologische interpretatie}
Temperatuur beïnvloedt niet alleen de fysiologie, maar ook migratie en voedselbeschikbaarheid. Warmer water duwt prooisoorten naar hogere breedtes en verlaagt de draagkracht van ecosystemen \cite{cheung2013}. Voor makohaaien kan dat leiden tot lagere groeisnelheden in tropische/subtropische kerngebieden en verschuivingen naar koelere watermassa's. Klimaatprojecties van het IPCC geven een verdere toename van 1{,}5--4~$^\circ$C deze eeuw \cite{ipcc2021}, waardoor temperatuurafhankelijk modelleren essentieel wordt voor beheer en vangstlimieten.
\newpage

%------------------------------------
% 4. BEPERKINGEN VAN DE UITBREIDING EN VERVOLGSTAPPEN
%------------------------------------
\section{Beperkingen van de uitbreiding en vervolgstappen}
Hoewel de uitbreiding inzicht geeft, gelden meerdere beperkingen:
\begin{itemize}
    \item \textbf{Q$_{10}$-waarden zijn soortspecifiek:} we gebruiken literatuurwaarden; soort- en leeftijdsspecifieke experimenten ontbreken.
    \item \textbf{Enkel temperatuur als externe factor:} voedselbeschikbaarheid, zuurstof (deoxygenatie) en oceaanverzuring blijven buiten beschouwing, terwijl deze factoren klimaatgedreven veranderen.
    \item \textbf{Scenariovereenvoudiging:} de simulaties veronderstellen een gelijkmatige temperatuurstijging en geen seizoens- of dieptevariatie, terwijl mako's juist sterk verticaal migreren.
    \item \textbf{Datakwaliteit:} de kalibratie steunt op één populatie (Zuid-Atlantisch); vergelijkbare datasets uit andere oceanen zijn nodig voor robuustheid.
\end{itemize}
Vervolgstappen zijn het koppelen van $T(z,t)$-profielen (diepte-afhankelijke temperatuur) aan de energiebalans, het modelleren van prey-indexen en het fitten van $Q_{10}$ direct op metabolische veldmetingen.
\newpage

%------------------------------------
% 5. CONCLUSIE
%------------------------------------
\section{Conclusie}
Het uitgebreid von Bertalanffy model laat zien dat temperatuurstijging zowel de groeisnelheid als de uiteindelijke massa van makohaaien beïnvloedt. Op basis van $Q_{10}$-schaalregels daalt $w^*$ met circa 12\% bij +2~$^\circ$C en 24\% bij +4~$^\circ$C, terwijl de tijd tot volwassen massa slechts beperkt versnelt. Deze resultaten onderstrepen dat klimaatverandering niet alleen verspreiding, maar ook individuele groei en daarmee populatiedynamiek van haaien raakt. Voor beheer betekent dit dat temperatuurscenario's expliciet in vangstadviezen en beschermingsplannen moeten worden meegenomen.
\newpage

%------------------------------------
% REFERENTIES
%------------------------------------
\begin{thebibliography}{99}
\addcontentsline{toc}{section}{Referenties}

\bibitem{bertalanffy1934}
von Bertalanffy, L. (1934). \textit{Untersuchungen über die Gesetzlichkeit des Wachstums}. Roux' Archiv für Entwicklungsmechanik der Organismen, 131, 613--652.

\bibitem{renner-martin2018}
Renner-Martin, K., Figueira, W. F., Buckley, L. B., \& Possingham, H. P. (2018). \textit{On the exponent in the von Bertalanffy growth model}. PeerJ, 6, e4205. \url{https://peerj.com/articles/4205/}

\bibitem{bertalanffy1957}
von Bertalanffy, L. (1957). \textit{Quantitative laws in metabolism and growth}. The Quarterly Review of Biology, 32, 217--231.

\bibitem{pardo2013}
Pardo, S. A. (2013). \textit{Avoiding fishy growth curves: Accounting for the influence of temperature and growth autocorrelation}. Methods in Ecology and Evolution, 4(2), 202--211. \url{https://doi.org/10.1111/2041-210X.12020}

\bibitem{gillooly2001}
Gillooly, J. F., Brown, J. H., West, G. B., Savage, V. M., \& Charnov, E. L. (2001). \textit{Effects of size and temperature on metabolic rate}. Science, 293(5538), 2248--2251.

\bibitem{cheung2013}
Cheung, W. W. L., Sarmiento, J. L., Dunne, J., Frölicher, T. L., Lam, V. W. Y., Palomares, M. L. D., Watson, R., \& Pauly, D. (2013). \textit{Shrinking of fishes exacerbates impacts of global ocean changes on marine ecosystems}. Nature Climate Change, 3, 254--258.

\bibitem{ipcc2021}
IPCC. (2021). \textit{Climate Change 2021: The Physical Science Basis}. Contribution of Working Group I to the Sixth Assessment Report of the Intergovernmental Panel on Climate Change. Cambridge University Press.

\bibitem{liu2021}
Liu, K.-M., et al. (2021). \textit{Multi‑Model Approach on Growth Estimation and Association With Life History Trait for Elasmobranchs}. Frontiers in Marine Science. 

\bibitem{smart2021}
Smart, J. J., \& Grammer, G. L. (2021). \textit{Modernising fish and shark growth curves with Bayesian length-at-age models}. PLoS ONE, 16(2), e0246734. \url{https://doi.org/10.1371/journal.pone.0246734}

\bibitem{araya2006}
Araya, M., \& Cubillos, L. A. (2006). \textit{Evidence for two-phase growth in elasmobranchs}. In J. A. Musick \& R. Bonfil (Eds.), Management techniques for elasmobranch fisheries (pp. 243--254). Springer. \url{https://doi.org/10.1007/978-1-4020-5570-6_9}

\bibitem{natanson2006}
Natanson, L. J., Casey, J. G., Kohler, N. E., \& Pratt, H. L. (2006). \textit{Age and growth of elasmobranch fishes}. In J. C. Carrier, J. A. Musick, \& M. R. Heithaus (Eds.), Biology of sharks and their relatives (pp. 211--225). CRC Press.

\bibitem{first_zd}
Sea first (z.d.). \textit{Haaien}. {{Sea First.}} Retrieved on 17 November 2025. \url{https://www.seafirst.org/haaien}

\bibitem{rolim2020}
Rolim, F. A., Crivelli, A. J., \& Oddone, M. C. (2020). \textit{Growth and derived life-history characteristics of the shortfin mako (Isurus oxyrinchus) in the South Atlantic Ocean}. Journal of Fish Biology, 97(3), 826--838. \url{https://doi.org/10.1111/jfb.14378}

\bibitem{cailliet2006}
Cailliet, G. M., Smith, W. D., Mollet, H. F., \& Goldman, K. J. (2006). \textit{Age and growth studies of chondrichthyan fishes: the need for consistency in terminology, verification, validation, and growth function fitting}. Environmental Biology of Fishes, 77(3-4), 211--228.

\end{thebibliography}
\newpage

%------------------------------------
% BIJLAGEN
%------------------------------------
\appendix
\section{Bijlage A: Python code}

De volledige Python code voor de analyses in dit artikel is beschikbaar in een Jupyter Notebook. Hieronder een samenvatting van de belangrijkste functies.

\subsection{Analytische oplossing}
\begin{lstlisting}[caption={Analytische oplossing von Bertalanffy}]
def analytical_solution(t, eta, kappa, w0):
    """
    Analytische oplossing van dw/dt = eta*w^(2/3) - kappa*w
    """
    w_inf_cbrt = eta / kappa  # w_inf^(1/3)
    w0_cbrt = w0**(1/3)
    return (w_inf_cbrt + (w0_cbrt - w_inf_cbrt) * 
            np.exp(-kappa * t / 3))**3
\end{lstlisting}

\subsection{Parameterschatting}
\begin{lstlisting}[caption={Kleinste kwadraten fitting}]
from scipy.optimize import minimize

def objective(params, t_data, w_data, w0):
    eta, kappa = params
    w_pred = analytical_solution(t_data, eta, kappa, w0)
    return np.sum((w_pred - w_data)**2)

result = minimize(objective, x0=[2.0, 0.3], 
                  args=(t_data, w_data, w0),
                  bounds=[(0.1, 10), (0.01, 1)])
\end{lstlisting}

\newpage

\section{Bijlage B: Extra Figuren}
\newpage

\end{document}
